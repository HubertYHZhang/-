\documentclass[11pt]{article}

\usepackage{amsmath,xeCJK,graphicx,setspace,geometry,natbib,indentfirst}

\author{陈昱东,王新淇,张嘉楠,张煜豪}
\title{\textbf{关于抖音电商和传统电商的定价对比的研究计划}}
\date{\today}
\geometry{left=1.0in,right=1.0in,top=1.0in,bottom=1.0in}

\onehalfspacing
\setlength{\parindent}{24pt}

\begin{document}

\maketitle

\section{导言}

\section{电商平台发展和运行概况}

\section{可能的差异与经济学理论支持}

由上一部分(可能需要上下接轨一下),抖音平台在几方面与传统的电商软件(平台)存在差异:一是推送的方式,二是直接面对的消费人群,三是运行的机制。(暂时瞎写)

\subsection{短视频与注意力缺陷}

注意力缺陷(Inattention)是经济学和心理学研究的重要话题,在这方面,一些行为经济学家做出了理论上的贡献。这当中,尤以“理性的注意力缺陷(Rational Inattention)”为代表,这一概念将注意力本身作为稀缺资源纳入经济模型中,并成为决策者内生决策决定分配的对象:给定消费者知道自己的注意力有限的情况下,他会主动将注意力分配到最值得分配的区域,从而使得他对不同的事物拥有不同的信息/信念(Belief)。\citet{moraga-gonzalezPricesHeterogeneousSearch2017}的理论模型描述了这样一种典型的状况:消费者

\section{实证检验}

\section{总结}

\clearpage
\section*{参考文献}
\bibliographystyle{aer}
\bibliography{theory}

\end{document}
